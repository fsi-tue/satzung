\section{Mitgliedschaft}
	\begin{myEnum}
		\item Ordentliches Mitglied können alle natürlichen Personen werden, die in einem der dem Fachbereich Informatik der Universität Tübingen angehörigen Studiengänge oder einem Lehramtsstudiengang mit Unterrichtsfach Informatik ordentlich immatrikuliert sind und die Ziele des Vereins unterstützen.
		\item Fördermitglied können alle natürlichen oder juristischen Personen werden, die die Ziele des Vereins unterstützen.
	\end{myEnum}

\section{Erwerb der Mitgliedschaft}
	\begin{myEnum}
		\item Die ordentliche Mitgliedschaft bzw. die Fördermitgliedschaft werden aufgrund einer schriftlichen Beitrittserklärung erworben, über die durch Beschluss des Vorstandes entschieden wird. Die Annahme ist schriftlich mitzuteilen.
		\item Im Fall der Ablehnung besteht ein Widerspruchsrecht. Über den Widerspruch entscheidet die nächste Mitgliederversammlung.
	\end{myEnum}

\section{Beendigung der Mitgliedschaft}
	\begin{myEnum}
		\item Die Mitgliedschaft endet,
			\begin{mySubEnum}
				\item wenn das Mitglied schriftlich gegenüber dem Vorstand seinen Austritt erklärt,
				\item wenn das Mitglied mit der Zahlung der Mitgliedsbeiträge zwei Jahre im Rückstand ist und der Vorstand
				daraufhin das Ende der Mitgliedschaft feststellt,
				\item wenn das Mitglied gegen die Satzung verstößt oder das Vereinsansehen schädigt und die Mitgliederversammlung daraufhin mit 3/4-Mehrheit den Ausschluss beschließt, oder
				\item mit dem Tod des Mitglieds.	
			\end{mySubEnum}
		\item Die ordentliche Mitgliedschaft geht in eine Fördermitgliedschaft über, wenn
			\begin{mySubEnum}
				\item das Mitglied in einem Zeitraum von zwei Jahren zu keiner Mitgliederversammlung erschienen ist oder
				\item falls die Bedingungen in §4 Ziff. 1 nicht mehr erfüllt sind.
			\end{mySubEnum}
		\item Das ordentliche Mitglied hat den Wegfall der Bedingungen dem Vorstand anzuzeigen.
	\end{myEnum}
	
\section{Mitgliedsbeiträge}
	\begin{myEnum}
		\item Von den ordentlichen Mitgliedern und den Fördermitgliedern kann ein jährlicher, ggf. nach Mitgliedsform differenzierter Beitrag erhoben werden, welcher unabhängig vom Beitrittstermin jeweils für das Kalenderjahr erhoben wird.
		\item Über die Höhe der Beiträge entscheidet die Mitgliederversammlung.
	\end{myEnum}

\section{Organe des Vereins}
	\begin{myEnum}
		\item Organe des Vereins sind:
			\begin{mySubEnum}
				\item die Mitgliederversammlung
				\item der Vorstand
			\end{mySubEnum}
	\end{myEnum}