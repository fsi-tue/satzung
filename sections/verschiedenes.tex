\section{Rechnungslegung, Kassenprüfung}
	\begin{myEnum}
		\item Der Vorstand hat der Mitgliederversammlung über die Kassenführung Rechnung zu legen und fertigt einen Jahresabschluss an.
		\item Mit dem Vorstand wählt jede Mitgliederversammlung für die gleiche Amtszeit zwei Vereinsmitglieder zu Kassenprüfenden. Sie prüfen zum Ende jedes Geschäftsjahres die Kassenführung und den Jahresabschluss	auf Richtigkeit und Vollständigkeit. Sie berichten der Mitgliederversammlung und beantragen die Entlastung des Vorstandes.
		\item Die Kassenprüfenden dürfen nicht Mitglieder des Vorstandes sein. Falls andere Kandidaten zur Verfügung stehen, dürfen sie nicht Mitglieder des Vorstandes des Vorjahres sein und nicht wiedergewählt werden
	\end{myEnum}
	
\section{Auflösung des Vereins}
	\begin{myEnum}
		\item Der Verein wird durch Beschluss seiner Mitgliederversammlung gemäß §9 Absatz 2 (VII) oder aus gesetzlichen Gründen aufgelöst. 
		\item Für die Auflösung des Vereins über die Mitgliederversammlung ist eine Anwesenheit von mindestens 50\% der Mitgliederschaft oder alternativ 6 ordentlichen Mitgliedern notwendig. Der Verein wird aufgelöst, falls eine Mehrheit von 3/4 der anwesenden Stimmberechtigten mit \textit{Ja} stimmt.
		\item Bei Auflösung des Vereins erfolgt keine Rückgewähr des Vereinsvermögens an die Mitglieder des Vereins.
		\item Bei Auflösung des Vereins oder bei Wegfall steuerbegünstigter Zwecke fällt das Vermögen an die Vereinigung der Freunde der Universität Tübingen e.V. (Universitätsbund), die es unmittelbar und ausschließlich für gemeinnützige Zwecke zu verwenden hat.
	\end{myEnum}