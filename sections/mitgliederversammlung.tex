\section{Zuständigkeiten}
	\begin{myEnum}
		\item Oberstes Organ ist die Mitgliederversammlung.
		\item Die Mitgliederversammlung ist zuständig für alle Angelegenheiten des Vereins, soweit sie nicht dem Vorstand zugewiesen sind. Im Einzelnen hat die Mitgliederversammlung u.a. folgende Aufgaben:
			\begin{mySubEnum}
				\item Wahl und Abwahl des Vorstands
				\item Entlastung des Vorstands
				\item Wahl der Rechnungsprüferenden des Vereins
				\item Entscheidung über den Widerspruch abgelehnter Bewerberinnen und Bewerber gemäß §5 Ziff. 2
				\item Entscheidung über die Erhebung von Mitgliedsbeiträgen und ggf. eine Beitragsordnung
				\item Änderungen der Satzung
				\item Auflösung des Vereins
			\end{mySubEnum}
	\end{myEnum}

\section{Turnus, Öffentlichkeit}
	\begin{myEnum}
		\item Die Mitgliederversammlung findet mindestens einmal im Jahr statt. 
		\item Jede Mitgliederversammlung findet öffentlich statt
	\end{myEnum}

\section{Einberufungen}
	Die Einladung zur Mitgliederversammlung wird den Mitgliedern vier Wochen vor der Mitgliederversammlung unter Angabe der Tagesordnung an die von ihnen angegebene E-Mail Adresse zugesandt.

\section{Beschlussfähigkeit}
	Die Mitgliederversammlung ist beschlussfähig, wenn sie frist- und formgerecht im Sinne von §11 einberufen wurde.
	
\section{Tagesordnung}
	\begin{myEnum}
		\item Jedes Mitglied kann bis zum Beginn der Mitgliederversammlung Anträge auf Ergänzung der Tagesordnung beim Vorstand stellen. Über die Annahme eines solchen Antrags entscheidet die Mitgliederversammlung durch Beschluss. Nicht Gegenstand eines Ergänzungsantrags können sein:
			\begin{mySubEnum}
				\item Auflösung des Vereins,
				\item Änderung der Satzung sowie
				\item Ausschluss eines oder mehrerer Mitglieder
			\end{mySubEnum}
		\item Einzelne Tagesordnungspunkte können auf Beschluss der Mitgliederversammlung unter Ausschluss der Öffentlichkeit behandelt werden.
		\item Die Tagesordnung jeder ordentlichen Mitgliederversammlung muss mindestens die folgenden Tagesordnungspunkte beinhalten:
			\begin{mySubEnum}
				\item Beschluss der Tagesordnung,
				\item Feststellung der Beschlussfähigkeit,
				\item Geschäftsbericht des Vorstandes,
				\item Bericht der Kassenprüfenden,
				\item Entlastung des Vorstandes sowie
				\item Wahl des Vorstandes.
			\end{mySubEnum}
	\end{myEnum}

\section{Leitung der Mitgliederversammlung}
	Die Leitung der Mitgliederversammlung wird vom Vorstand festgelegt.
	
\section{Stimmrecht}
	Jedes anwesende ordentliche Mitglied ist in der Mitgliederversammlung einfach stimmberechtigt. Fördermitglieder sind nicht stimmberechtigt.
	
\section{Beschlussfassung}
	\begin{myEnum}
		\item Bei Beschlussfassungen kann jedes anwesende, stimmberechtigte Mitglied mit \textit{Ja}, \textit{Nein} oder \textit{Enthaltung} stimmen. Zur Beschlussfassung genügt, falls nicht anders geregelt, eine Mehrheit der \textit{Ja}- über den \textit{Nein}-Stimmen. Bei Stimmengleichheit ist der Antrag abgelehnt.
		\item Bei der Wahl des Vorstandes oder der Kassenprüfenden werden, falls mehr Kandidierende zur Wahl stehen, als es Ämter zu besetzen gilt, die drei bzw. zwei Kandidierenden, die die meisten Stimmen auf sich vereinigen	können, gewählt. Jedes abstimmende Mitglied hat so viele Stimmen, wie Ämter zu besetzen sind.
		\item Abstimmungen erfolgen stets geheim. Davon kann abgewichen werden, falls kein stimmberechtigtes Mitglied Einspruch erhebt, jedoch niemals, wenn Gegenstand der Abstimmung die Besetzung eines Amtes ist.
		\item Das Nähere kann eine Geschäftsordnung regeln.
	\end{myEnum}

\section{Protokoll}
	\begin{myEnum}
		\item Ein vom Vorstand benanntes Vereinsmitglied fertigt ein Protokoll der Mitgliederversammlung an.
		\item Das Protokoll gibt Aufschluss über die Ergebnisse von Abstimmungen.
		\item Das Protokoll ist binnen sieben Tagen den Mitgliedern bekannt zu machen.
		\item Den  Vereinsmitgliedern  bleiben  dreißig  Tage  nach  der  Bekanntmachung  des  Protokolls  zum  Stellen  von Änderungsanträgen, über die der Vorstand entscheidet. Wird das Protokoll dadurch nachträglich geändert, muss das Protokoll erneut entsprechend obiger Bestimmungen bekannt gemacht werden.
	\end{myEnum}

\section{Außerordentliche Mitgliederversammlung}
	\begin{myEnum}
		\item Der Vorstand veranstaltet eine außerordentliche Mitgliederversammlung, wenn
			\begin{mySubEnum}
				\item die Belange des Vereins es erfordern,
				\item die Schatzmeister*innen es für erforderlich halten oder
				\item mindestens ein Drittel der Mitglieder gemeinsam unter Angabe der Tagesordnung bei dem Vorstand eine solche beantragen.
			\end{mySubEnum}
		\item Die außerordentliche Mitgliederversammlung findet zum frühsten möglichen Zeitpunkt statt. Die in §11 bestimmten Vorschriften zur Einberufung sind davon unberührt.
	\end{myEnum}