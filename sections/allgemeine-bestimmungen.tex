\section{Name und Sitz}
	\begin{myEnum}
		\item Der Verein führt den Namen \textit{Fachschaft Informatik Tübingen} (kurz: \textit{fsi Tübingen}).
		\item Er hat seinen Sitz in Tübingen. 
		\item Der Verein ist nicht im Vereinsregister eingetragen.
	\end{myEnum}
	
\section{Ziele und Aufgaben des Vereins}
	\begin{myEnum}
		\item Der Verein verfolgt ausschließlich und unmittelbar gemeinnützige Zwecke im Sinne des Abschnitts „Steuerbegünstigte Zwecke“ der Abgabenordnung.
		\item Das Ziel des Vereins ist die Förderung der Bildung und der Studentenhilfe.
		\item Der Satzungszweck wird verwirklicht durch:
			\begin{mySubEnum}
				\item Betreuung von Studienanfängern.
				\item Bereitstellung von Lehrmaterialien (z.B. Skripte und Altklausuren) und Arbeitsutensilien (z.B. Präparierbestecke).
				\item Ansprechpartner für studentische Belange.
				\item Anbieten einer kostenfreien Infrastruktur mit Mailling-Listen.
				\item Betreiben einer Homepage mit vereinsbezogenen Informationen
			\end{mySubEnum}
	\end{myEnum}

\section{Selbstlosigkeit}
	\begin{myEnum}
		\item Der Verein ist selbstlos tätig; er verfolgt nicht in erster Linie eigenwirtschaftliche Zwecke.
		\item Mittel des Vereins dürfen nur für die satzungsmäßigen Zwecke verwendet werden.
		\item Es darf keine Person durch Ausgaben, die dem Zweck des Vereins fremd sind, oder durch unverhältnismäßig hohe Vergütungen begünstigt werden.
		\item Die Mitglieder erhalten in ihrer Eigenschaft als Mitglied keine Zuwendungen aus Mitteln des Vereins. Sie haben bei ihrem Ausscheiden keinerlei Ansprüche an das Vereinsvermögen.
	\end{myEnum}
